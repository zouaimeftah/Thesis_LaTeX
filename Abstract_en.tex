
\thispagestyle{plain}
%\addcontentsline{toc}{section}{Abstract}
\section*{Abstract}

In antiquity, human beings created shelters of woods, stones, and mud to protect themselves from natural factors and predatory animals. Then they developed them according to their needs and available materials until they got to their current shape.
Nowadays, houses became equipped with a lot of electronic devices, and the Internet has become indispensable in all homes. These caused humans to think of creating fully automated homes to provide a safer and more comfortable lifestyle. This is where the term as \textbf{smart homes} was introduced.

The smart house is a house that is fully controlled by a group of remote-control devices (smartphones and computers) and managed through the Internet or a wireless local network. With the advent of  \textbf{Internet of Things} and \textbf{cloud computing} concepts, smart homes have taken on new features such as interactive smart houses. The components of these houses became more independent in decision-making, intelligent, and interactive with the environment through sensors and built-in microelectronic devices.


Through this thesis, we will study the problems of communication between the components of houses and the Internet, the interaction between these components, and the changes that occur in the surrounding environment automatically through the use of multi-agents (mobile and fixed) system. We will present a new architecture for the objects (things) by adding the  \textbf{agent layer} to their base architecture. This layer gives them the ability to communicate, cooperate, coordinate, and share information and knowledge. It also grants them \textbf{independence} in decision-making and a degree of \textbf{intelligence}. The \textbf{collective intelligence distributed} among the objects creates an intelligent system capable of making correct and independent decisions.
