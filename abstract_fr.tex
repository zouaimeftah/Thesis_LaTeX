
\thispagestyle{plain}
\addcontentsline{toc}{section}{Résumés}
\section*{Résumé}
A l'antiquité, l'homme a créé des huttes (maisons d'arbres et de peau) pour le protéger des facteurs naturels et des animaux prédateurs pour répondre aux ses besoins de sécurité, puis ces huttes se sont développées en fonction des besoins humains et matériaux disponibles de construction jusqu'à l'apparition des maisons actuels. Les maisons de nos jours sont devenues trop équipées des appareils électroniques et électriques avec  l'émergence des machines domotiques, et avec l'immense diffusion de l’internet, avec tous ces facteurs technologiques et infrastructures, les besoins humains ont évolué et il est devenu possible de créer des maisons entièrement automatisées qui fournissent un environnement plus confortable et plus sûr. le couplage de ces technologies avec l'intelligence artificielle se donné naissance à un nouveau modèle de maison dite \textbf{maison intelligente}.


La maison intelligente est une maison entièrement contrôlée et gérée par la technologie à travers un groupe de serrures et de dispositifs de télécommande (smartphones et ordinateurs). La maison est entièrement gérée via Internet ou un réseau local sans fil. Avec l'avènement du concept de l'\textbf{Internet des objets} et du \textbf{cloud computing}, les maisons intelligentes ont adopté de nouvelles fonctionnalités telles que : \textbf{la maison intelligente interactive}, et le contenu de ces maisons est devenu une prise de décision indépendante, une intelligence et une interaction avec l'environnement via des capteurs et dispositifs microélectroniques intégrés.


Dans cette thèse, nous avons étudié le problème de la communication entre les composants de la maison afin qu'ils ne se connectent pas seulement à Internet mais aient également la capacité de communiquer entre eux et d'interagir automatiquement avec les changements qui se produisent dans l'environnement. Grâce à l'utilisation d'un système multi-agents (mobiles et située), où nous avons présenté une nouvelle architecture  pour les objets en ajoutant \textbf{la couche d'agent } à la architecture de base de l'objet. Cette couche donne à l'objet la capacité de communiquer, de coopérer, coordonner et de partager des informations et des connaissances. Elle lui confère un \textbf{degré d'intelligence} et \textbf{d'indépendance de décision},\textbf{ l'intelligence collective répartie} entre les objets crée un système intelligent capable de prendre des décisions correctes et indépendantes.
