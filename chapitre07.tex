\chapter*{Conclusion générale et perspectives}
\rhead{\itshape Conclusion générale et perspectives}
\mark{CConclusion générale et perspectives}
\addcontentsline{toc}{chapter}{Conclusion générale et perspectives}
\section*{Conclusion}

La nouvelle informatique se caractérise par des nouveaux domaines qui représentent la tendance de la technique comme le cloud computing et l'internet des objets et le big data prenant une grande partie de cette nouvelle informatique. Le Big Data représente un changement significatif dans les technologies de l'information. Techniquement, nous vivons un véritable phénomène de rupture. En effet, au-delà de quelques dizaines de téraoctets, les technologies traditionnelles sont inadéquates, elles ne permettent plus d'analyser la volumétrie élevée et disparate des données (le 3V du Big Data),et la question qui se pose alors est de savoir comment et ou se traiter cette quantité de donnee,  au niveau de l'objet ou cloud,chaqu'une a des avantages et les inconvénients et des limites.

Dans cette thèse, nous avons décrit en détail nos contrubtions: 

Notre première contribution consiste en la proposition  d'une nouvelle approche basée agents pour smart house. Nous avons utilisé la plate-forme Jade dans laquelle nous avons lancé nos agents. Cette plate-forme  est utilisée dans les différents implémentations pour heberger des agents, nous avons utilisé le protocole MQTT  également pour transmettre les données entre les objets (edges), Fog et le cloud.

Dans la deuxième contribution, nous avons présenter une architecture IoT autonome, c'est une nouvelle architecture pour l'IoT en utilisant le paradigme agent. Nous avons profité des fonctionnalités de l'agent pour enrichir l'appareil de qualités d'autonomie, de coordination et de coopération. Nous avons implémenté cette approche en utilisant une Raspberry Pi avec intégration des capteurs et en utilisant le protocole MQTT pour les communications.

Dans  la troisième contribution, nous avons proposé un Robot IoT basé sur la deuxième contribution. Ce robot est doté d’une  caméra routable et des capteurs intégrés. Son rôle principal est de  percevoir l’environnement à la demande des objets, autrement, il offre un service de sensing (sensing as service).

Pour la quatrième et la derniére contribution, nous avons  crée  une API qui simule une maison intelligente, cette maison intelligente consiste à connecter les différents appareils et systèmes de la maison afin qu’ils puissent être contrôlés de n’importe où et provoquer l’interaction souhaitée entre eux. Ces appareils sont des objets connectés via un réseau qui a une fenêtre sur Internet. 

Toutes ces contributions ont été implémentées en utilisant la plateforme JADE pour manipuler les agents et les langages JAVA et JAVAFX comme outils de programmations. 

Pour concrétiser nos solutions, nous avons utilisé les cartes éléctroniques telles que : Raspberry PI, Arduino et des capteurs.



\section*{Perspectives}
Dans le futur et comme perspectives de ce travail, nous considérerons les extensions suivantes:
\begin{itemize}
   \item Faire un couplage avec le modéle de Big-Data; 
   \item exploiter la technologie Cloud Robotics pour les smart cities;
   \item Orienter nos prochains travaux vers la sécurité des maisons intelligentes;
   \item Développer une strategie basée sur nos contributions pour la sécurité dans une smart city;
   \item faire une projection de nos résultats pour les vehicules autonomes;

\end{itemize}
